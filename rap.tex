\documentclass{article}
\usepackage[utf8]{inputenc}
\begin{document}

\section{fonctionnement du gestionnaire E/S}

\textit{Le gestionnaire E/S est composée de plusieurs fonctionnalité}

\subsection{verifier le format}
\begin{verbatim}
les fonctions verifier_format_etat_alphabet(),verifier_format_transition()
se charge de la fonctionalité verifier_format,
lorsque l'utilisateur charge un fichier,ces fonctions verifie si le format 
du fichier respecte le format prédéfinie 
\end{verbatim}

\subsection{charger fichier}
\begin{verbatim}
la fonction charger_fichier() se charge de la fonctionalité charger fichier,
l'interface envoie le nom du fichier à la fonction chargé fichier
qui l'ouvre pour extraire les données et les enregistrer 
 dans les variables de la structures T_machine,
les variable globales(=NC,NE,NR,NT) et remplir la matrice de transition
\end{verbatim}

\subsection{sauvegarde d'un fichier}
\begin{verbatim}
la fonction conversion_donne_fichier() se charge de la fonctionalité 
sauvegarde fichier,
l'interface envoie des données et le nom du fichier
 où on va sauvegarder les données à la fonction conversion_donne_fichier.
cette fonction cree le fichier à partir du nom du fichier reçu par l'interface 
et enregistre les données reçu par l'interface
(=qui correspond à  la structure machine_info) dans le fichier
\end{verbatim}
\subsection{creation des logs}
\begin{verbatim}
les fonctions creation_log(),ecriture_log(T_machine T,FILE* fichier1,
FILE* fichier2,FILE* fichier3,int *position_texte),fermeture_log() se charge
de la fonctionalité création des logs
- la fonction création_log() va ouvrir 3 fichier
1 fichier en latex
1 fichier en postscript
1 fichier  en xfig
- la fonction ecrire_log ecrire chaque transition effectué par le programme
-la fonction fermeture_log ferme les fichier_log

\end{verbatim}
\end{document}
